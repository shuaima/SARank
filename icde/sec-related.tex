%\stitle{Related work}.

\section{Related work} \label{sec-related}

%We summarize related work as follows.


Scholarly article ranking has shifted from citation count analysis~\cite{Garfield471,Hirsch15112005} to graph analysis~\cite{ChenXMR07,Zhou07-CoRank,Jiang12-MRank,Liang16AAAI,Li08TSRanking,Wang13AAAI,WalkerXKM07,sayyadi09,
Wang16TIST,Ng11KDD}.
Based on the information used, these methods are divided into four categories: (a) using the citation information only~\cite{Garfield471,Hirsch15112005,ChenXMR07,Ng11KDD}, (b) using the citation and temporal information~\cite{Li08TSRanking,WalkerXKM07}, (c) using the citation information and other heterogeneous information, \eg authors and venues of articles~\cite{Zhou07-CoRank,Jiang12-MRank,Liang16AAAI}, and (d) combining the citation, temporal and other heterogeneous information~\cite{sayyadi09,Wang16TIST,Wang13AAAI}.
Our work belongs to the last category aiming at fully employing information available for scholarly article ranking.
%
\marked{Moreover, recent work has also leveraged external data to improve the ranking quality, \eg using Knowledge Graph Embedding to better understand the meaning of research concepts~\cite{XiongPCWWW17}, and has explored scientific journal ranking \cite{PackalenB17} and scholar ranking \cite{ZhangNBKZX17}. Different from these, our work ranks articles based on scholarly data only.}


%\stitle{PageRank\&weighted PageRank algorithms}.

%PageRank \cite{Brin98:PageRank} and its extensions have been extensively used for citation analyses \cite{Waltman2014}. While PageRank equally propagates scores along outlinks, Weighted PageRank \cite{Xing04:WPR} extends PageRank by distributing scores based on the popularity of pages. Different from previous work, the Time-Weighted PageRank proposed in this work discriminately propagates scores in terms of citation statistics.

PageRank \cite{Brin98:PageRank} and its extensions have been extensively used for citation analyses \cite{Waltman2014}. While PageRank equally propagates scores along outlinks, Weighted PageRank extends PageRank by distributing scores based on certain criteria such as popularity of pages~\cite{Xing04:WPR} or authority of authors~\cite{Ding11}. \marked{Both of them fail to capture the time-dependent characteristics, a key factor for scholarly article ranking. There has been work extending temporal information into PageRank, \eg exponential decayed weights~\cite{Li08TSRanking}, exponential decayed initial vector~\cite{WalkerXKM07} and time-dependent weights based on co-authorship~\cite{FIALA2012370}.} Different from previous work, the Time-Weighted PageRank of this work discriminately propagates scores in terms of citation statistics.






%\stitle{Dynamic algorithms}.

Dynamic algorithms have proven useful for various tasks by avoiding computing from scratch~\cite{RamalingamR93}.
% and only recomputing those affected by updates
%Dynamic algorithms have proven useful for graph analysis tasks, \eg incremental graph pattern matching~\cite{FanWW13} and  incremental simrank computation~\cite{YuLZ14}.
To our knowledge, little concern has been paid to dynamic scholarly article ranking except that~\cite{GhoshKHLL11} uses PageRank in dynamic citation networks. However, its solution is based on a strong and impractical assumption that there are no citations between articles in the same years.
Further, although there exist several studies on incremental PageRank computation~\cite{DesikanPSK05,AbiteboulPC03,WuR09} and on incremental PageRank approximation \cite{BahmaniCG10,BahmaniKMU12}, they are not designed for scholarly article ranking.
%
Different from previous work, we study scholarly article ranking in a dynamic environment in terms of
the citation characteristics of scholarly articles, which has never been exploited before.

%Our approach only makes the assumption that there are no mutual references within the citation network, which, we admit, violates xx\% of total citations on \magdata, and is significantly different (yy\% on \magdata) from~\cite{GhoshKHLL11}.  - move to Section 3

Ensemble methods use multiple learners to obtain better performance than could be obtained from a constituent learner alone~\cite{zhihua-book}.
%In this work, we leverage ensembles to produce better and robust results for scholarly article ranking~\cite{zhihua-book,wsdmcup,DuanAMHH16}.
In this work, we leverage  importance assembling  to produce better and more robust results for scholarly article ranking~\cite{zhihua-book,wsdmcup,DuanAMHH16}.
