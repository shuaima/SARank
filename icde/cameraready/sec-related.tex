%\stitle{Related work}.

\section{Related work} \label{sec-related}

%We summarize related work as follows.


Scholarly article ranking has shifted from citation-count analysis~\cite{Garfield471,Hirsch15112005} to graph analysis~\cite{Waltman2014,Jiang12-MRank,Liang16AAAI,Li08TSRanking,Wang13AAAI,WalkerXKM07,sayyadi09,
Wang16TIST,Ng11KDD}.
Based on the information used, these methods are divided into four categories: (a) using the citation information only~\cite{Garfield471,Hirsch15112005,Ng11KDD}, (b) using the citation and temporal information~\cite{Li08TSRanking,WalkerXKM07}, (c) using the citation information and other heterogeneous information, \eg authors and venues of articles~\cite{Jiang12-MRank,Liang16AAAI}, and (d) combining the citation, temporal and other heterogeneous information~\cite{sayyadi09,Wang16TIST,Wang13AAAI}.
Our work belongs to the last category aiming at fully employing information available for scholarly article ranking.
%
%Moreover, recent work has also leveraged external data to improve the ranking quality, \eg using Knowledge Graph Embedding to better understand the meaning of research concepts~\cite{XiongPCWWW17}, and has explored scientific journal ranking \cite{PackalenB17} and scholar ranking \cite{ZhangNBKZX17}. Different from these, our work ranks articles based on scholarly data only.


%\stitle{PageRank\&weighted PageRank algorithms}.

PageRank \cite{Brin98:PageRank} and its extensions have been extensively used for citation analyses \cite{Waltman2014}. While PageRank equally propagates scores along outlinks, Weighted PageRank extends PageRank by distributing scores based on certain criteria such as popularity of pages~\cite{Xing04:WPR} or authority of authors \cite{Ding11}.
%Both these approaches fail to capture the time-dependent characteristics, a key factor for scholarly article ranking.
Scholarly graphs belong to temporal graphs~\cite{temporalgraph}, and temporal information is a key factor for scholarly article ranking.
There has been work extending temporal information into PageRank, \eg exponentially decayed weights~\cite{Li08TSRanking}, exponentially decayed initial vectors~\cite{WalkerXKM07} and time-dependent weights based on co-authorship~\cite{FIALA2012370}.
%Different from previous work,
Differently, our Time-Weighted PageRank is designed based on a deep analysis of scholarly articles, and discriminately propagates scores in terms of citation statistics.


%\stitle{Dynamic algorithms}.
Dynamic algorithms have proven useful for various tasks by avoiding computing from scratch~\cite{RamalingamR93}.
To our knowledge, little concern has been paid to dynamic scholarly article ranking except that~\cite{GhoshKHLL11} uses PageRank in dynamic citation networks. However, its solution is based on a strong and impractical assumption that there are no citations between articles in the same years.
Further, although there exist several studies on incremental PageRank computation~\cite{DesikanPSK05,AbiteboulPC03,WuR09} and on incremental PageRank approximation \cite{BahmaniCG10,BahmaniKMU12}, they are not designed for scholarly article ranking.
%
In this work, we study dynamic scholarly article ranking in the general setting by eliminating the strong and impractical assumption. Our incremental algorithm is designed for the block-wise algorithm of Time-Weighted PageRank, and is based on the citation characteristics, both of which have never been exploited before.
%Different from previous work, we study scholarly article ranking in a dynamic environment in terms of the citation characteristics of scholarly articles, which has never been exploited before.


Ensemble methods use multiple learners to obtain better performance than could be obtained from a constituent learner alone~\cite{zhihua-book}.
In this work, we leverage  importance assembling  to produce better and more robust ranking for scholarly articles~\cite{zhihua-book,wsdmcup,DuanAMHH16}.
