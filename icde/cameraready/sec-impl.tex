\section{Dealing with Missing Data}
\label{sec-impl}

Data quality is one of the most challenging issues in large scale data management, especially for data from open domains and multiple sources, \eg the Microsoft Academic Graph (MAG)~\cite{Sinha15:MAG}.
The early version of MAG has $120$ million scholarly articles, among which we find that there are about $73$ million articles without references and about $77$ million ones without venues. The ranks of those articles with missing information are underestimated by our model \ewpr, since ensembles assign the minimum scores to articles. As a result, data missing seriously impairs the ranking accuracy.

As for references and venues, the later are easier to obtain, and each filled venue can have a direct and substantial impact on the article ranking, \ie $R_v(u)$ of Eq.~(\ref{eq-ensemble}). In contrast, a filled reference only has an indirect and slight impact. Hence, we decide to use external data to fill in missing venues.


%\subsection{Data Collecting}
%\label{subsec-datacraw}
\stitle{Data collecting}.
The raw external data is collected from publicly available Digital Libraries, such as IEEE Xplore ({\footnotesize http://ieeexplo-re.ieee.org/gateway/}),  PubMed ({\footnotesize http://www.ncbi.nlm.nih.gov/pub-med/}) and DBLP ({\footnotesize http://dblp.uni-trier.de/db/}). In total, we collect $2.8$ million articles with venue information as our external data, in which there are $57,000$ different venues.
 
 

%\subsection{Data Preprocessing}
%\label{subsec-dataprep}
\stitle{Data preprocessing}.
The venues in MAG are well processed, and are replaced by their series names. For example, {\em ``9th International Conference on Web Search and Data Mining, 2016''} is replaced with {\em ``Web Search and Data Mining''}. This makes it hard to directly link with the collected raw venue names. Hence, we preprocess raw venue names for the simplification of subsequent venue linking.
%
We first remove stop words such as {``on''} and common words like {``Conference''}, as well as years and some special characters from collected raw venue names. Then the same venues are merged, and the number of different venue names is reduced to $42,000$.

%\subsection{Data Linking}
%\label{subsec-datamap}
\stitle{Data linking}.
The final and also the most important step of filling missing venue information is to link each collected venue name to an existing one in MAG. Intuitively, linking based on name similarity is the most effective way such that two venues are linked if their names bear high similarity. We exploit the Jaro metric to evaluate the name similarity, which is based on the number and order of the common characters between two strings, and obtains good results in tasks such as record linkage and name matching~\cite{Cohen03strcompa}. Formally, a collected venue name is linked to an existing one in MAG if their Jaro similarity exceeds a pre-define threshold.

However, such a threshold is nontrivial to determine in practice. A high threshold can guarantee the accuracy of linked pairs, while only a tiny proportion of collected venue names are linked. On the other hand, a low threshold increases the number of linked pairs, which, in the same time, also introduces many errors. In order to reach a good balance between the number of linked pairs and the accuracy, we propose to combine another constraint on topic similarity of venues for linking, and only weaker filter conditions need be used in both constraints.

In MAG, fields of study (FOS) represent research topics of articles, such as {\em Web pages}, and {\em language technology}. Hence, we use FOS to evaluate the topic similarity of two venues. There are about $54,000$ FOS in MAG and most articles are assigned with two or three FOS. Let the set of FOS of each venue be the union of the sets of FOS of articles published in that venue. And the topic similarity of two venues based on FOS is defined as:
\vspace{-1ex}

\begin{small}
\begin{equation} \label{eq-fos}
TS(s,t)=({|F_s\bigcap F_t|})/{\sqrt{|F_s|\cdot|F_t|}},
\end{equation}
\end{small}

\vspace{-1ex}
\noindent in which $s$ and $t$ are two venues, and $F_s$ and $F_t$ are the sets of FOS of $s$ and $t$, respectively.

When we link a collected venue name, it is directly linked to the most similar one in terms of name similarity, if their Jaro similarity exceeds a high threshold $\lambda$. Otherwise, we first use the topic similarity constraint to select several candidates in MAG, \ie venues whose topic similarities with the collected venue exceed a threshold $\theta$. Intuitively, these candidates are in the similar fields of the collected one. We then select the most similar candidate in terms of name similarity as its linked venue, if their Jaro similarity exceeds another threshold $\phi$. Hence, the collected venue is linked to the one to which it is similar in terms of both topics and names.

In our model \ewpr, threshold $\lambda$ is set to $0.95$, while thresholds $\theta$ and $\phi$ need not be very high, which are $0.5$ and $0.7$, respectively. Finally, $6,000$ among the $42,000$ collected venues are linked, resulting in $340,000$ (about $12\%$) articles with enriched venue information. Note that a majority of the collected venue names are not valid venues, such as booktitles and names of workshops, and cannot be linked to any one in MAG.
