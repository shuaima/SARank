\documentclass{letter}
\usepackage{geometry}

% duan
\usepackage{xspace}

\geometry{left=2.0cm, right=2.0cm, top=2.5cm, bottom=2.5cm}
\newcommand{\ie}{\emph{i.e.,}\xspace}
\newcommand{\eg}{\emph{e.g.,}\xspace}
\newcommand{\wrt}{\emph{w.r.t.}\xspace}
\newcommand{\aka}{\emph{a.k.a.}\xspace}
\newcommand{\kwlog}{\emph{w.l.o.g.}\xspace}
\newcommand{\etal}{\emph{et al.}\xspace}
\newcommand{\sstab}{\rule{0pt}{8pt}\\[-2.4ex]}



\begin{document}

Prof. xxx,\\
Editor-in-Chief,\\
IEEE International Conference on Data Engineering\\

Dear Prof. xxx,

Attached please find a revised version of our submission to IEEE International Conference on Data Engineering, \emph{Query Independent Scholarly Article Ranking}.

The paper has been substantially revised according to the referees�� comments. In particular, (1) we have further clarified our contributions, (2) we have explained the author component better, (3) we have added figures to show the statistics of data, (4) we have reported the memory cost in experience, (5) we have added related work of time delaying problem, and (6) we have also taken this opportunity to rewrite several parts of the paper to improve the presentation.

We would like to thank all the referees for their thorough reading of our paper and for their valuable comments.

Below please find our responses to the comments by the referees.

%******************* reviewer 1 ***********************************************
\line(1,0){500}

\textbf{Response to the comments of Referee 1.}

\textbf{[R1W1]} \emph{Formula 7 in SARank still needs parameters, which are not easy to set.}

The parameters are designed to fit variety ranking with different demands. For example, the ground-truth RECOM prefer venue component while PFCTN prefer citation component. Furthermore, our parameters are quite flexible to choose in a certain range for suiting different ranking which we have discussed in Exp-4.3.

\textbf{[R1W2]} \emph{The extension of this method to query dependent ranking or other heterogeneous information network should be discussed.}

We focus on query independent ranking because of its application in Google Schorlar and MicroSoft Academic Search. As to extension of query dependent ranking, our ranking scores can combine with the relevance scores given a specific query. For other heterogeneous information network, the extension of our method is not known.


\textbf{[R1D1]} \emph{In the time-weighted pagerank algorithm, the impacts are encoded into the edge weight. How about changing the values on nodes?}

Actually, changing the values on nodes is the key of our ranking method which evaluate the weight of nodes. Moreover, the change of nodes is difficult to handle due to the multiple iterations and probably cause the result fail to converge. To our knowledge, there is no related work on changing node weight and it is unclear how to change it. Finally, we set the weight of edges to achieve the change of impacts since they are essentially propagable in the network through edges.


\textbf{[R1D2]} \emph{This paper has an important observation that ��the impact of an article tends to decay with time after the peak time only.�� However, I cannot see that from Figure 1. It is better to add some references to support the claim.}

Figure 1 shows the logscale number of citations grow exponentially with time, instead of impacts of individual articles.



\textbf{[R1D3]} \emph{How to determine the peak time for the PeakMul pattern?}

The PeakMul pattern is defined in [19]:  the height of a peak should be at least 75\% of the maximum peak height, and two consecutive peaks should be separated by more than two years. In our work, we have simplified the pattern and only use the latest peak time which we have discussed in section II-A.


[19]  T. Chakraborty, S. Kumar, P. Goyal, N. Ganguly, and A. Mukherjee, ��On the categorization of scientific citation profiles in computer science,�� Commun. ACM, vol. 58, no. 9, 2015.


\textbf{[R1D4]} \emph{For large scholar graph, such as MAG, it is better to use the distributed graph processing framework to improve the scalability of the method}

Yes, we agree that distributed processing is an important and independent problem which will be studied in future work and we have discussed it in Section VII Conclusion. Thanks for the suggestion!


\textbf{[R1D5]} \emph{TWPageRank works better on the scholar graphs than the general web graph. is it due to that the citation graph is an directed acyclic graph?}

Yes, but the citation graph does have cycles and is not a DAG. Compare to general web graphs, the computation of TWPageRank in scholarly graphs is more efficient as a result of the less SCC edge ratio which is shown in Table I. In addition, we have designed the incremental algorithm based on this feature of scholalry graphs and it works better.

\textbf{[R1D6]} \emph{MAG is used before description in Page 1. Some cells in Table 1 have two values, while others have one value. The meaning should be given explicitly}

Yes, we have revised as suggested. Thanks for the suggestion!


%******************* reviewer 2 ***********************************************
\line(1,0){500}

\textbf{Response to the comments of Referee 2.}

\textbf{[R2W1]} \emph{While practically the paper makes a good case about the particular application domain and its contributions there, the paper lacks technical depth.}

We think that the techniques proposed also have their own interests: (1) An ensemble based ranking model itself is hard to define, (2) an incremental algorithm that makes use of the particular properties of the scholar graph, and (3) a through experimental study. That is, this paper certainly has its contributions to the technical part.


\textbf{[R2W2]} \emph{Presentation could improve a bit since the paper has too much notation making it hard to read/follow.}

Yes, we have gone over the paper and improved the presentation to make it more understandable. Thanks for the suggestion!



\textbf{[R2D1]} \emph{There seem to be two "contradictory" sides to this paper. From one side, this is a great "applications" paper where existing techniques and some good intuition is used to get better results into an existing (practical) problem. From a purely research and technical perspective, the paper is on the weak side. Pagerank and (incremental) versions of it already exist and the paper makes little contributions there. The claim (made twice) in the related work section that "Different from previous work" this paper focuses on citations statistics doesn't hold water. The fact that the authors are considering a different application domain is not a contribution compared to past (incremental) pagerank papers. Unless there is a better claim to be made about the algorithmic side of the paper, as things stand, the paper makes minimal technical contributions.}

We thank the reviewer for appreciating the practice merit of our work. And we believe this paper also stands from research and technique perspective. To our knowledge, the incremental algorithm of time-weighted PageRank has not gained enough attention and is lack of research. Furthermore, the exist incremental PageRank for dynamic scholarly article ranking which mentioned in [22] is based on a impractical assumption. Therefore, we have designed the incremental time-weighted PageRank algorithm for scholarly article ranking which has its unique features.  We have further clarified our contributions in Section I Introduction and claimed the technique difference of our algorithm in Section VI Related Work.

[22]  R. Ghosh, T. Kuo, C. Hsu, S. Lin, and K. Lerman, ��Time-aware ranking in dynamic citation networks,�� in ICDM Workshops, 2011.


\textbf{[R2D2]} \emph{The abstract starts with the sentence "Ranking query independent scholarly articles is a critical and challenging task, due to the heterogeneous, evolving and dynamic nature of entities involved in scholarly articles". If three types of data (articles, authors and venues) that are updated (maybe) weekly is considered challenging in terms of heterogeneity and dynamicity then one must wonder what one can say about real-life scenarios with thousands of data types where hundreds/thousands of new items are inserted daily (consider any online merchant system) and were ranking is important (recommendations). In summary, this sentence should go. Not to mention that the task is not really "critical".}

Yes, we have downgraded the tone and changed it into "practical and difficult". Thanks for the suggestion!


\textbf{[R2D3]} \emph{On page 4, you say "However, the resulting author citation graph to compute the prestige is typically too large to handle". This is a questionable statement. Maybe it is too large to handle "on a PC with 2 Intel Xeon E5�C2630 2.4GHz CPUs and 64 GB of memory" but approaches like PageRank has been used for web-scale data so unless there is something that I am missing, this statement should also go (or explained better).}

We have removed the sentence and explained the statement better. We have added the explanation of the generation of author citation graph and the size of the graph which makes it computationally expensive.

%******************* reviewer 3 ***********************************************
\line(1,0){500}

\textbf{Response to the comments of Referee 3.}


\textbf{[R3W1\&D2]} \emph{More examples are suggested, especially when introducing the definitions, to make it easier for readers to understand.}

Yes, we agree that there should be more examples and add figures to illustrate how to generate components from scholarly data. However, due to the space limitation, we choose to report the examples in extended version.

\textbf{[R3W2\&D4]} \emph{The memory cost of the algorithms is better to be reported in the experiments.}

We have added the report of the memory cost in experiment Exp-x.x by numbers instead of figures due to the space limitation.Thanks for the suggestion!

\textbf{[R3D1]} \emph{Although the authors have a detailed introduction about the growing data, detailed figures about the data are recommended in the introduction.}

Yes, we have added a figure x to show the volume statistics of the data.

\textbf{[R3D3]} \emph{It is suggested that the authors should include some references to papers addressing the time delaying problem such as temporal graph, since it is related to the topic.}

We have added citations [xx][yy] about time delaying problem and discussed it in Section VI Related Work.

\line(1,0){500}

Your sincerely,

Shuai Ma, Chen Gong, Renjun Hu, Dongsheng Luo, Chunming Hu, Jinpeng Huai

\end{document}
