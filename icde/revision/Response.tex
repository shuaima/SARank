\documentclass{letter}
\usepackage{geometry}

% duan
\usepackage{xspace}
\usepackage{color}

\newcommand{\kw}[1]{{\ensuremath {\mathsf{#1}}}\xspace}
\newcommand{\PairAcc}{\kw{PairAcc}}
\newcommand{\recom}{{\sc Recom}\xspace}
\newcommand{\fcita}{{\sc PFCtn}\xspace}
\newcommand{\pagerank}{\kw{PRank}} %PageRank
\newcommand{\futurerank}{\kw{FRank}} %FutureRank
\newcommand{\hhgrank}{\kw{HRank}} %HHGBiRank
\newcommand{\ensemblerank}{\kw{SARank}}
\newcommand{\marked}[1]{\textcolor{red}{#1}}
\newcommand{\magdata}{{\sc MAG}\xspace}
\newcommand{\aminer}{{\sc DBLP}\xspace}

\geometry{left=2.0cm, right=2.0cm, top=2.5cm, bottom=2.5cm}
\newcommand{\ie}{\emph{i.e.,}\xspace}
\newcommand{\eg}{\emph{e.g.,}\xspace}
\newcommand{\wrt}{\emph{w.r.t.}\xspace}
\newcommand{\aka}{\emph{a.k.a.}\xspace}
\newcommand{\kwlog}{\emph{w.l.o.g.}\xspace}
\newcommand{\etal}{\emph{et al.}\xspace}
\newcommand{\sstab}{\rule{0pt}{8pt}\\[-2.4ex]}



\begin{document}

%Prof. xxx,\\
%Editor-in-Chief,\\
%IEEE International Conference on Data Engineering\\

Dear Meta-Reviewer and Reviewers,

We have revised our submission, \emph{Query Independent Scholarly Article Ranking}, to IEEE International Conference on Data Engineering 2018.

We would like to thank all the referees for their thorough reading of our paper and for their valuable comments.

The paper has been substantially revised according to the referees�� comments. In particular, we have (1) further clarified our contributions, (2) added figures to better illustrate the data growing, (3) explained the author component better,  (4) reported the memory cost in experiments, (5) modified related work to differentiate our technique with existing ones, (6) answered the questions raised by referees, and (7) taken this opportunity to rewrite several parts of our paper to improve the presentation.

Below please find our detailed responses to the comments by the referees.


\line(1,0){500}

\textbf{Response to the comments of Meta-Reviewer.}

\textbf{[M1]} \emph{The reviewers see some merit in the approach, but they have also raised important concerns regarding the technical depth and novelty of the approach. In particular, the authors use known techniques for an application (scientific article ranking) for which there is abundance of previous work (it even has its own field, called scientometrics, or  bibliometrics).}




We do believe that there are technical contributions: (1) An ensemble based ranking model (a great effort needs to be paid to develop such a ranking model);  (2) An incremental algorithm designed for dynamic scholarly article ranking in the general setting, which is designed for the block-wise algorithm of Time-Weighted PageRank, and is based on the citation characteristics. These have never been exploited in existing studies. That is, we are not simply using known techniques for an application.



We also believe that our ensemble based ranking model and the citation statistics revealed in our study have their own interests to the researchers in the field of scientometrics or  bibliometrics.

 We have clarified these in our paper. Thanks for pointing these out!

\textbf{[M2]} \emph{In revising their paper, the authors should try to address all reviewers�� comments and in particular address the required changes for revision by Reviewer 1.}

Yes, we have addressed these. Please refer to our responses below.


%******************* reviewer 1 ***********************************************
\line(1,0){500}

\textbf{Response to the comments of Referee 1.}

\textbf{[R1W1]} \emph{Formula 7 in SARank still needs parameters, which are not easy to set.}

We discuss the necessity and the selection of parameters right after Eq. (7) (left column, page 4). More specifically, these parameters make our model be able to fit to the various ranking scenarios. As shown in the experiment study (Exp-1 and Exp-2), our model performs well in two reasonable ranking scenarios by using quite different aggregating parameters.


Moreover, these parameters are indeed quite flexible to choose within a certain range (Exp-5.3, page 10). Note that (a) the optimal \PairAcc is obtained within a single region, (b) the \PairAcc keeps at a high level within a certain ($\alpha$, $\beta$) combination space around the optimal region, and (c) the optimal parameters on the same set of ground-truth are very similar across different datasets.


%The parameters are designed to fit variety ranking with different demands. For example, the ground-truth RECOM prefer venue component while PFCTN prefer citation component. Furthermore, our parameters are quite flexible to choose in a certain range for suiting different ranking which we have discussed in Exp-4.3.

\textbf{[R1W2]} \emph{The extension of this method to query dependent ranking or other heterogeneous information network should be discussed.}

%We discuss the extension of query dependent ranking in the revised paper (left column, page 2).
%
This work focuses on {\em query independent ranking} because of its wide application in modern scholarly search engines such as Google Scholar, Semantic Scholar and Microsoft Academic Search (as we discussed in the first two paragraphs in Section 1).
%
The combination of query dependent ranking and query independent ranking  is well known, e.g., typical search engines such as Google : keyword search (query dependent ranking) is combined with Pagerank (query independent ranking). Our ranking can serve as an indicator of the importance of items, and, given a specific query, it can either be directly used to rank a subset of relevant items or combine with any relevance scores for ranking along the same lines as keyword search.


The extension in {\em other heterogeneous information networks} is currently unclear, since the TWPageRank is designed based on a deep analysis of scholarly articles (left column, page 2). We shall identify the possible applications in other heterogeneous information networks in the future.


\textbf{[R1D1]} \emph{In the time-weighted pagerank algorithm, the impacts are encoded into the edge weight. How about changing the values on nodes?}

%We clarify the reasons of not changing values on nodes at the revised version (Remarks, left column, page 3).
The goal of our TWPageRank is to change node values, \ie importance, through iterations.
%One may wonder why not directly change node values.
An alternative way to do that is to directly change node values. However, it is difficult to handle as importance does not have a clear and closed-form definition. It may also cause problems for convergence.
To our knowledge, there is no related work on changing node weights to achieve this goal, and it is indeed unclear how to handle it. % Finally, we use iterations with impact weights since importance is essentially propagable through edges and it is also fair for all nodes.


\textbf{[R1D2]} \emph{This paper has an important observation that ��the impact of an article tends to decay with time after the peak time only.�� However, I cannot see that from Figure 1. It is better to add some references to support the claim.}

Figure 1 shows the logscale percentage of the number of articles/citations of each year that grows exponentially with time, instead of the number of citations (\ie impacts) of individual articles.
%We have highlighted it in the revised version (Figure 1, page 2).

We have explained our claim better (Time-Weighted PageRank, page 2). The direct decay model only considers the case of {\em MonDec}.
To unify these distinct citation patterns and to make our ranking model succinct, we adopt the claim that the impact of an article tends to decay with time after the peak time.
Moreover, the aging function in [17] exploits a log-normal survival probability that also only decays with time after the citation peak.
%(it increases before reaching the peak since it aims to predict the number of citations instead of impacts of articles).

[17]  D. Wang, C. Song, and A. Barab��si, ��Quantifying long-term scientific impact,�� Science, vol. 342, no. 6154, pp. 127-132, 2013.



\textbf{[R1D3]} \emph{How to determine the peak time for the PeakMul pattern?}

The {\em PeakMul} pattern is originally defined in [18]:  the height of a peak should be at least 75\% of the maximum peak height, and two consecutive peaks should be separated by more than two years. In our work, we unify the five patterns, and we only use the highest peak time that has been formally defined in Section II-A (para 1, left column, page 3).
Also note that: (a) [18] allows multiple peaks to exist for an article while we only consider the highest peak, and (b) [18] uses the original numbers of citations to compute peaks, and we use the scaled numbers of citations \wrt the total number of citations of all articles at each year.


[18]  T. Chakraborty, S. Kumar, P. Goyal, N. Ganguly, and A. Mukherjee, ��On the categorization of scientific citation profiles in computer science,�� Commun. ACM, vol. 58, no. 9, 2015.


\textbf{[R1D4]} \emph{For large scholar graph, such as MAG, it is better to use the distributed graph processing framework to improve the scalability of the method}

Yes, we agree that distributed processing is an important way for large scholar graphs.  But this topic is beyond the reach of this study, and we plan to do this in our future work (Conclusions, page 12). Thanks for the suggestion!


\textbf{[R1D5]} \emph{TWPageRank works better on the scholar graphs than the general web graph. is it due to that the citation graph is an directed acyclic graph?}

This work does NOT assume the DAG structure of scholarly graphs, where cycles are allowed to exist (Observation, left column, page 5).
Compared with general Web graphs, the computation of TWPageRank on scholarly graphs is more efficient as a result of the {\em temporal order of citations}, which gives less SCC edge ratio as shown in Table I (page 5). In addition, we further design incremental algorithms based on the temporal order of citations, \eg for the maintenance of the topological order, which also makes the computation more efficient.
%We have highlighted the statements about temporal order in the revised version.

\textbf{[R1D6]} \emph{MAG is used before description in Page 1. Some cells in Table 1 have two values, while others have one value. The meaning should be given explicitly.}

Yes, we have revised our paper as suggested (right column, page 1 \& Table I, page 5). Thanks!
%Thanks for the suggestion!


%******************* reviewer 2 ***********************************************
\line(1,0){500}

\textbf{Response to the comments of Referee 2.}

\textbf{[R2W1]} \emph{While practically the paper makes a good case about the particular application domain and its contributions there, the paper lacks technical depth.} \\
\textbf{[R2D1]} \emph{There seem to be two ``contradictory" sides to this paper. From one side, this is a great ``applications" paper where existing techniques and some good intuition is used to get better results into an existing (practical) problem. From a purely research and technical perspective, the paper is on the weak side. Pagerank and (incremental) versions of it already exist and the paper makes little contributions there. The claim (made twice) in the related work section that ``Different from previous work" this paper focuses on citations statistics doesn't hold water. The fact that the authors are considering a different application domain is not a contribution compared to past (incremental) pagerank papers. Unless there is a better claim to be made about the algorithmic side of the paper, as things stand, the paper makes minimal technical contributions.}

We thank the reviewer for appreciating the practice merit of our work which is drawn from an important and practical problem.

%And we believe this paper also stands from research and technique perspective.
And We think that the techniques proposed also have their own interests: a) An importance assembling based ranking model itself is hard to define and tune, b) an incremental algorithm for block-wise time-weighted PageRank computation that makes use of the particular properties of the scholarly graphs, and c) a thorough experimental study. That is, this paper certainly has its contributions to the technical part.

More specifically, the only existing dynamic scholarly article ranking solution [21] is based on an impractical assumption that there is no citations between articles of the same years.
In this work, we study dynamic scholarly article ranking in the general setting by eliminating the strong and impractical assumption. Our incremental algorithm is designed for the block-wise algorithm of Time-Weighted PageRank, and is based on the citation characteristics (\eg for the maintenance of the topological order), both of which have never been exploited before.
%
To conclude, the incremental time-weighted PageRank algorithm for scholarly graphs has its own unique features, which is not simply applying existing techniques into a different domain.
We have further %clarified our contributions in Introduction and
claimed the technique difference of our algorithm in Related Work.

%The property of scholarly data makes the incremental computation of (time-weighted) PageRank quite different from those on general graphs, which, to our knowledge, has not gained enough attentions. Furthermore, the only existing dynamic scholarly article ranking solution [21] is based on an impractical assumption that there is no citations between articles of the same years.

%\marked{In the revised version, we highlight our technical contributions.}

[21]  R. Ghosh, T. Kuo, C. Hsu, S. Lin, and K. Lerman, ��Time-aware ranking in dynamic citation networks,�� in ICDM Workshops, 2011.

\textbf{[R2W2]} \emph{Presentation could improve a bit since the paper has too much notation making it hard to read/follow.}

Yes, we have gone over the paper and improved the presentation to make it more understandable.
%Thanks for the suggestion!


\textbf{[R2D2]} \emph{The abstract starts with the sentence ``Ranking query independent scholarly articles is a critical and challenging task, due to the heterogeneous, evolving and dynamic nature of entities involved in scholarly articles". If three types of data (articles, authors and venues) that are updated (maybe) weekly is considered challenging in terms of heterogeneity and dynamicity then one must wonder what one can say about real-life scenarios with thousands of data types where hundreds/thousands of new items are inserted daily (consider any online merchant system) and were ranking is important (recommendations). In summary, this sentence should go. Not to mention that the task is not really ``critical".}

We have downgraded the tone and changed it into ``practical and difficult". Thanks for the suggestion!


\textbf{[R2D3]} \emph{On page 4, you say ``However, the resulting author citation graph to compute the prestige is typically too large to handle". This is a questionable statement. Maybe it is too large to handle ``on a PC with 2 Intel Xeon E5�C2630 2.4GHz CPUs and 64 GB of memory" but approaches like PageRank has been used for web-scale data so unless there is something that I am missing, this statement should also go (or explained better).}

We have removed the sentence and explained more details about the statement (Author Component, left column, page 4).
%
One way to compute the prestige of authors is to construct an author citation graph  such that (a) a node represents an author, and (b) a direct edge $(s,t)$ denotes that there exist articles of author $s$ citing articles of author $t$. However, it is easy to see that for each citation, the corresponding two sets of authors are fully connected, which makes it computationally expensive to compute the prestige of authors on such an author citation graph with TWPageRank.
%
Indeed, the generated author citation graph has 7.12 billion edges on the largest \magdata, \ie the memory cost of storage only is 57GB. Please also refer to our new experiments on memory costs.

%Moreover specifically, we add the details of the generation of the author citation graph. The generated graph has 7.12 billion edges on the largest \magdata (\ie the memory cost of storage only is 57GB), which makes it computationally expensive to compute prestige of authors  with TWPageRank on that graph. We hence turn to the alternative light-weight method to achieve the goal.

%******************* reviewer 3 ***********************************************
\line(1,0){500}

\textbf{Response to the comments of Referee 3.}


\textbf{[R3W1\&D2]} \emph{More examples are suggested, especially when introducing the definitions, to make it easier for readers to understand.}

Yes, thanks for the nice suggestion! Due to the space limitation, we have added the examples to the full version (https://hurenjun.github.io/SARank-full.pdf).

\textbf{[R3W2\&D4]} \emph{The memory cost of the algorithms is better to be reported in the experiments.}

We have reported the memory costs in a new set of experiments, \ie Exp-4 (page 10). Thanks for the suggestion!
%that on the largest \magdata, (\futurerank, \ensemblerank) cost (16.42GB, 49.42GB) memory when $Y_s=2015$, and \hhgrank costs 61.37GB memory when $Y_s=2014$. Interestingly, our \ensemblerank model costs more memory than \futurerank due to its model complexity, while it is more efficient with our batch and incremental algorithms. seperate exp, between two baselines, author citation graph
%the memory cost of (\pagerank, \futurerank, \ensemblerank) is (12.65GB, 16.42GB, 49.42GB) when $Y_s=2015$, and the one of \hhgrank is 61.37GB when $Y_s=2014$.


\textbf{[R3D1]} \emph{Although the authors have a detailed introduction about the growing data, detailed figures about the data are recommended in the introduction.}

Yes, we  have added the volume statistics (the number of articles published at each year) of the growing data in Figure 1. Thanks!

\textbf{[R3D3]} \emph{It is suggested that the authors should include some references to papers addressing the time delaying problem such as temporal graph, since it is related to the topic.}

Yes. We have discussed the correlation between our work and temporal graphs, and cited necessary references (e.g., P. Holme and J. Saramaki, ��Temporal networks,�� Physics Reports, vol. 519, no. 3, pp. 97 �C 125, 2012) in related work (Section VI).

\line(1,0){500}

Your sincerely,

Shuai Ma, Chen Gong, Renjun Hu, Dongsheng Luo, Chunming Hu, Jinpeng Huai

\end{document}
