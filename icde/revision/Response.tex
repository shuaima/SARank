\documentclass{letter}
\usepackage{geometry}

% duan
\usepackage{xspace}
\usepackage{color}

\newcommand{\kw}[1]{{\ensuremath {\mathsf{#1}}}\xspace}
\newcommand{\PairAcc}{\kw{PairAcc}}
\newcommand{\recom}{{\sc Recom}\xspace}
\newcommand{\fcita}{{\sc PFCtn}\xspace}
\newcommand{\marked}[1]{\textcolor{red}{#1}}

\geometry{left=2.0cm, right=2.0cm, top=2.5cm, bottom=2.5cm}
\newcommand{\ie}{\emph{i.e.,}\xspace}
\newcommand{\eg}{\emph{e.g.,}\xspace}
\newcommand{\wrt}{\emph{w.r.t.}\xspace}
\newcommand{\aka}{\emph{a.k.a.}\xspace}
\newcommand{\kwlog}{\emph{w.l.o.g.}\xspace}
\newcommand{\etal}{\emph{et al.}\xspace}
\newcommand{\sstab}{\rule{0pt}{8pt}\\[-2.4ex]}



\begin{document}

%Prof. xxx,\\
%Editor-in-Chief,\\
%IEEE International Conference on Data Engineering\\

Dear Reviewers,

This is the response letter regarding the revised version of our submission to IEEE International Conference on Data Engineering 2018, \emph{Query Independent Scholarly Article Ranking}.

The paper has been substantially revised according to the referees�� comments. In particular, we have (1) further clarified our contributions (Section I), (2) added figures to show the statistics of data (3) explained the author component better,  (4) reported the memory cost in experiments, (5) added related work of time delaying problem, and (6) taken this opportunity to rewrite several parts of our paper to improve the presentation.

We would like to thank all the referees for their thorough reading of our paper and for their valuable comments.

Below please find our detailed responses to the comments by the referees.

%******************* reviewer 1 ***********************************************
\line(1,0){500}

\textbf{Response to the comments of Referee 1.}

\textbf{[R1W1]} \emph{Formula 7 in SARank still needs parameters, which are not easy to set.}

We discuss the necessity and the selection of parameters right after Eq. (7) (left column, page 4). More specifically, these parameters enable our model to fit to the various ranking scenarios in real-life applications, which hardly can be covered with one set of parameters. As be shown in the experiments, our model performs well in two reasonable ranking scenarios (with \recom and \fcita) by using quite different aggregating parameters, and, moreover, these parameters are indeed quite flexible to choose within a certain range, \ie $[0,1]$, as a) the optimal \PairAcc is obtained within a single region, b) the \PairAcc keeps at a high level within a certain ($\alpha$, $\beta$) combination space around the optimal region, and c) the optimal parameters on the same set of ground-truth are very similar across different datasets (please also refer to Exp-4.3).

%The parameters are designed to fit variety ranking with different demands. For example, the ground-truth RECOM prefer venue component while PFCTN prefer citation component. Furthermore, our parameters are quite flexible to choose in a certain range for suiting different ranking which we have discussed in Exp-4.3.

\textbf{[R1W2]} \emph{The extension of this method to query dependent ranking or other heterogeneous information network should be discussed.}

This work focuses on query independent ranking because of its wide application in modern scholarly search engines such as Google Scholar, Semantic Scholar and Microsoft Academic Search.
%
We discuss the extension of query dependent ranking in the revised paper (left column, page 2), that our ranking can serve as an indicator of the importance of items, and, given a specific query, it can either be directly used to rank a subset of relevant items or combine with any relevance scores for ranking.
%
The extension in other heterogeneous information networks is currently unclear, since the TWPageRank is derived from the analysis of impacts of articles (left column, page 2). We shall identify such possible applications in the future.


\textbf{[R1D1]} \emph{In the time-weighted pagerank algorithm, the impacts are encoded into the edge weight. How about changing the values on nodes?}

We clarify the reasons of not changing values on nodes at the revised version (Remarks, left column, page 3).
The goal of our TWPageRank is to change node values, \ie importance.
%One may wonder why not directly change node values.
Directly changing node values, however, is difficult to handle as importance does not have a clear and closed-form definition and it may also cause problems for convergence.
To our knowledge, there is no related work on changing node weights and it is yet unclear how to handle it. Finally, we use impact weights to achieve the goal since importance is essentially propagable through edges and it is also fair for all nodes.


\textbf{[R1D2]} \emph{This paper has an important observation that ��the impact of an article tends to decay with time after the peak time only.�� However, I cannot see that from Figure 1. It is better to add some references to support the claim.}

Figure 1 shows the logscale percentage of the number of citations of each year which grows exponentially with time (due to the exponential growth of volume of  scientific publications), instead of numbers of citations (\ie impacts) of individual articles.
We have highlighted it in the revised version (Figure 1, page 2).
\marked{Moreover, we add references [x,y] to support our observation.}



\textbf{[R1D3]} \emph{How to determine the peak time for the PeakMul pattern?}

The PeakMul pattern is originally defined in [19]:  the height of a peak should be at least 75\% of the maximum peak height, and two consecutive peaks should be separated by more than two years. In our work, we unify the five patterns and only use the latest peak time which we have discussed in Section II-A (para 1, left column, page 3).


[19]  T. Chakraborty, S. Kumar, P. Goyal, N. Ganguly, and A. Mukherjee, ��On the categorization of scientific citation profiles in computer science,�� Commun. ACM, vol. 58, no. 9, 2015.


\textbf{[R1D4]} \emph{For large scholar graph, such as MAG, it is better to use the distributed graph processing framework to improve the scalability of the method}

Yes, we agree that distributed processing is an important while independent problem which we have already placed in our future work (Conclusions, page 12). %Thanks for the suggestion!


\textbf{[R1D5]} \emph{TWPageRank works better on the scholar graphs than the general web graph. is it due to that the citation graph is an directed acyclic graph?}

Yes. This work does NOT assume the DAG structure of scholar graphs, where cycles are allowed to exist (Observation, left column, page 5).
Compared to general web graphs, the computation of TWPageRank on scholarly graphs is more efficient as a result of the temporal order of citations, which gives less SCC edge ratio as shown in Table I (page 5). In addition, we further design incremental algorithms based on the temporal order, which also makes the computation more efficient.
%We have highlighted the statements about temporal order in the revised version.

\textbf{[R1D6]} \emph{MAG is used before description in Page 1. Some cells in Table 1 have two values, while others have one value. The meaning should be given explicitly}

Yes, we have revised our paper as suggested (right column, page 1 \& right column, page 5). Thanks for the suggestion!


%******************* reviewer 2 ***********************************************
\line(1,0){500}

\textbf{Response to the comments of Referee 2.}

\textbf{[R2W1]} \emph{While practically the paper makes a good case about the particular application domain and its contributions there, the paper lacks technical depth.}

We think that the techniques proposed also have their own interests: a) An ensemble based ranking model itself is hard to define and tune, b) an incremental algorithm that makes use of the particular properties of the scholarly graphs, and c) a through experimental study. That is, this paper certainly has its contributions to the technical part.
%\marked{In the revised version, we highlight our technical contributions.}


\textbf{[R2W2]} \emph{Presentation could improve a bit since the paper has too much notation making it hard to read/follow.}

\marked{Yes, we have gone over the paper and improved the presentation to make it more understandable. }
%Thanks for the suggestion!



\textbf{[R2D1]} \emph{There seem to be two ``contradictory" sides to this paper. From one side, this is a great ``applications" paper where existing techniques and some good intuition is used to get better results into an existing (practical) problem. From a purely research and technical perspective, the paper is on the weak side. Pagerank and (incremental) versions of it already exist and the paper makes little contributions there. The claim (made twice) in the related work section that "Different from previous work" this paper focuses on citations statistics doesn't hold water. The fact that the authors are considering a different application domain is not a contribution compared to past (incremental) pagerank papers. Unless there is a better claim to be made about the algorithmic side of the paper, as things stand, the paper makes minimal technical contributions.}

We thank the reviewer for appreciating the practice merit of our work which is drawn from an important and practical problem.
%
And we believe this paper also stands from research and technique perspective (Please also refer to our response to R2W1).
%
The property of scholarly data makes the incremental computation of (time-weighted) PageRank quite different from on general graphs, which, to our knowledge, has not gained enough attentions. Furthermore, the only existing dynamic scholarly article ranking solution [22] is based on a impractical assumption that there is no citations between articles of the same years.
%
Therefore, we have designed the incremental time-weighted PageRank algorithm for scholarly graphs which has its own unique features.  To conclude, we are not simply applying existing techniques into a different domain.
%
We have further clarified \marked{our contributions in Introduction} and claimed the technique difference of our algorithm in Related Work.
%


[22]  R. Ghosh, T. Kuo, C. Hsu, S. Lin, and K. Lerman, ��Time-aware ranking in dynamic citation networks,�� in ICDM Workshops, 2011.


\textbf{[R2D2]} \emph{The abstract starts with the sentence "Ranking query independent scholarly articles is a critical and challenging task, due to the heterogeneous, evolving and dynamic nature of entities involved in scholarly articles". If three types of data (articles, authors and venues) that are updated (maybe) weekly is considered challenging in terms of heterogeneity and dynamicity then one must wonder what one can say about real-life scenarios with thousands of data types where hundreds/thousands of new items are inserted daily (consider any online merchant system) and were ranking is important (recommendations). In summary, this sentence should go. Not to mention that the task is not really "critical".}

We have downgraded the tone and changed it into ``practical and difficult". Thanks for the suggestion!


\textbf{[R2D3]} \emph{On page 4, you say "However, the resulting author citation graph to compute the prestige is typically too large to handle". This is a questionable statement. Maybe it is too large to handle "on a PC with 2 Intel Xeon E5�C2630 2.4GHz CPUs and 64 GB of memory" but approaches like PageRank has been used for web-scale data so unless there is something that I am missing, this statement should also go (or explained better).}

We have removed the sentence and explained the statement better. Moreover specifically, we add the details of the generation of the author citation graph \marked{with and 115 billion edges 7.12 billion edges} which makes it computationally expensive (Author Component, left column, page 3).

%******************* reviewer 3 ***********************************************
\line(1,0){500}

\textbf{Response to the comments of Referee 3.}


\textbf{[R3W1\&D2]} \emph{More examples are suggested, especially when introducing the definitions, to make it easier for readers to understand.}

Yes, we agree that more examples (especially on how to generate components from scholarly data) will make our paper easier to understand. Due to the space limitation, we choose to add the examples in the extended version.

\textbf{[R3W2\&D4]} \emph{The memory cost of the algorithms is better to be reported in the experiments.}

\marked{We have added the memory cost in experiment Exp-3. Indeed, the memory cost of (PRank, FRank, SARank) is (12.65GB, 16.42GB, 49.42GB) while set the $Y_s=2015$ on MAG, respectively, and the memory cost of HRank is while set the $Y_s=2014$ on MAG. }

\textbf{[R3D1]} \emph{Although the authors have a detailed introduction about the growing data, detailed figures about the data are recommended in the introduction.}

Yes, \marked{we add the volume statistics of the growing data in Figure 1.}

\textbf{[R3D3]} \emph{It is suggested that the authors should include some references to papers addressing the time delaying problem such as temporal graph, since it is related to the topic.}

Yes. We have discussed the correlation between our work and temporal graph and include necessary references in Related Work.

\line(1,0){500}

Your sincerely,

Shuai Ma, Chen Gong, Renjun Hu, Dongsheng Luo, Chunming Hu, Jinpeng Huai

\end{document}
