\section{Ranking Computation}
\label{sec-alg}

In this section, we present our batch algorithm for computing scholarly article ranking based on our  model \ensemblerank.

%\marked{We first introduce the framework of our batch algorithm, and then explain the details.}

\subsection{Algorithm Framework}
\label{subsec-bat-alg}

Our batch algorithm \batensemble  combines the importance scores computed by the citation, venue and author components with Eq.~(\ref{eq-ensemble}). It takes as input academic graph data $D$ and an iteration threshold $\epsilon$ and returns the scholarly article ranking of $D$. It first constructs the citation and venue graphs $G^c(V^c,E^c)$ and $G^v(V^v,E^v)$. Then it computes the prestige and popularity of citation, venue and author components.
Finally, it combines the  prestige and popularity of the three components to produce the final ranking with Eq.~(\ref{eq-ensemble}).



%\stitle{Popularity computation}. It is easy to see that
For popularity computation, it is easy to see that
(a) the popularity of articles can be computed by scanning through all citations once and adding the freshness of citations to their corresponding articles, by Eq.~(\ref{eq-pop}), and (b) the popularity of venues in a specific year or authors is computed by averaging the popularity of the articles published in the venues or by the authors.
That is, the popularity computation can be done by scanning through all citations once.


%\stitle{Prestige computation}. As the prestige of authors is defined as the average prestige of their published articles,
For prestige computation, as the one of authors is defined as the average prestige of their published articles,
it suffices to scan through all author-article  relationships for computing the prestige of authors. The prestige of  articles and venues in a specific year is computed by TWPageRank on citation graphs and venue graphs, which is usually computed in an iterative manner~\cite{Brin98:PageRank} and is the most expensive computation. Hence, the key of the computation of our approach is a good solution for computing TWPageRank.


%While graphs derived from scholarly data is a special class of graphs in the sense that edges strictly follow {\em temporal order}, \eg an article can only cite others that are published earlier. In this section, we exploit this temporal order of scholarly data in ensemble computation, which significantly speeds up the efficiency.




\subsection{TWPageRank Computation}
\label{subsec-TWPageRank-computation}

The main result here is to speed up computing TWPageRank by exploiting the {\em temporal order} of scholarly articles.


\begin{claim}
\label{thm-prestige}
%TWPageRank can be computed in linear time in practice on scholarly data obeying a temporal order.
A block-wise PageRank computation method~\cite{Berkhin05} is a good choice for TWPageRank on scholarly data.
\end{claim}

The main idea of the block-wise PageRank computation is that each strongly connected component (\scc) of the input graph is treated as a block, and blocks are processed one by one following the {\em topological order} of the block-wise graph, \ie each node represents a block of the original graph~\cite{Berkhin05}.
We next show Claim~\ref{thm-prestige} by introducing and analyzing such a block-wise computation method.



%the directed and acyclic structure of the  can improve the efficiency of PageRank computation~\cite{Berkhin05,ObjectRank04}.

\stitle{Block-wise algorithm~\twprscc}.  It takes as input a citation or venue graph $G$ and an iteration threshold $\epsilon$, and returns the TWPageRank vector of $G$.
Moreover, to process an \scc instead of the entire graph, the edges of the citation or venue graph $G(V, E)$ are partitioned into the sets $E_i$ and $E_a$ of edges inside and across \sccs  such that $E_i\cap E_a = \emptyset$ and $E = E_i \cup E_a$.  The update rule in Eq.~(\ref{eq-twpr-update})  is revised accordingly to treat $E_i$ and $E_a$ separately as follows.

\vspace{-1ex}
\begin{scriptsize}
\begin{equation}\label{eq-prscc}
%\begin{split}
PR(v) =  d \sum_{(u,v)\in E_i} M_{u,v} PR(u)
   + d \sum_{(u,v)\in E_a} M_{u,v} PR(u) +  \frac{1-d}{n}.
%\end{split}
\end{equation}
\end{scriptsize}
%
\noindent
It first computes the block-wise graph $G'$ by treating \sccs in $G$ as single nodes, and then derives a topological order $O$ of nodes in $G'$. It then processes each \scc in the topological order  $O$ with Eq.~(\ref{eq-prscc}). Finally, it returns the TWPageRank vector. When processing an \scc, it iteratively updates the TWPageRank scores of the nodes in the \scc, and the iteration continues until the sum of TWPageRank score changes is less than $\epsilon\cdot\frac{|scc|}{|V|}$, where $|scc|$ is the number of nodes in the \scc. Note that there must exist a topological order $O$, as the block-wise graph is a directed acyclic graph~\cite{CormenLRS01}.


\eat{
%%% algorithm details
%%%%%%%%%%%%%%%%%%%%%Algorithm
\begin{figure}[tb!]
%\vspace{2ex}
\begin{center}
{\small
\begin{minipage}{3.36in}
\myhrule \vspace{-1.5ex}
\mat{0ex}{
%%%%%%%%%%%%%%%%%%%
{\sl Input:\/} \= citation graph or venue graph $G(V,E)$, iteration threshold $\epsilon$.\\
{\sl Output:\/} the TWPageRank vector of $G$. \\
\bcc \hspace{1.5ex}\=  $G'$ := the converted graph of $G$; \\
\icc \>  $O$ := topological order of $G'$; \\
\icc\>  \For each node $v'$ following $O$ \Do\\
\icc\>\hspace{2.5ex}\=  $scc$ := the corresponding SCC of $v'$; \\
\icc\>\> \While sum of changes of $PR(v)$ where $v\in scc$ exceeds $\frac{|scc|}{|V|} \epsilon$ \Do \\
\icc\>\>\hspace{3.5ex}\= update $PR(v)$ using Eq.~(\ref{eq-prscc}) where $v\in scc$; \\
\icc\> \Return $PR$.}
\vspace{-2.5ex} \myhrule
\end{minipage}
}
\end{center}
\vspace{-3ex}
\caption{\small Algorithm~\twprscc for computing TWPageRank} \label{alg-TWPageRank-venue}
\vspace{-3ex}
\end{figure}

%%%%%%%%%%%%Algorithm
}

%\stitle{Algorithm~\twprscc}.
%The block-wise algorithm to compute the TWPageRank vector for citation or venue graph $G$ is shown in Fig.~\ref{alg-TWPageRank-venue}.
%
%It first computes the converted graph $G'$  by treating SCCs in $G$ as single nodes (line 1), and then derives a topological order $O$ of $G'$ (line 2). It then iteratively updates the TWPageRank scores of nodes in each SCC in the topological order of $G'$ using  Eq.~(\ref{eq-prscc}) (lines 3--6), and finally returns the TWPageRank vector (line 7). More specifically, the iteration continues until the sum of TWPageRank score changes is less than $|scc|\cdot\epsilon/|V|$ (lines 5--6). Note that there must exist at least one topological order of $G'$ since $G'$ is a DAG.



\begin{cor} \label{prop-prscc}
The vector $PR$ returned by~\twprscc converges such that $||PR-PR^*||_1 < \epsilon$ where vector $PR^*$ is the convergent TWPageRank vector~\cite{Berkhin05}.
\end{cor}

\begin{proofSketch}
We first prove that the sum of changes after another iteration from $PR$, \ie $||d M^T PR + (1-d)e/n-PR||_1$, is smaller than $\epsilon$, and then prove that $||PR^*-PR||_1$ is smaller than the sum of changes. %(see~\cite {ERank-full} for details). % since $\epsilon > ||PR^+-PR||_1 = ||d\cdot M^T \cdot (PR-PR^*)||_1 + ||PR-PR^*||_1$.
\end{proofSketch}


%%% detailed proof sketch
\eat{
(1) We first prove that the sum of changes of TWPageRank vector after another iteration is less than $\epsilon$, \ie $||PR^+-PR||_1 < \epsilon$ where $PR^+=d\cdot M^T\cdot PR + \frac{1-d}{n}\cdot e$.
(2) Let $PR^-$ be the TWPageRank vector in the previous iteration of each SCC, and $PR_k$, $PR_k^-$ and $PR_k^+$ be the TWPageRank vectors of nodes within the $k$-th SCC in $PR$, $PR^-$ and $PR^+$, respectively. Note that SCCs are ordered such that the sequence of the corresponding nodes of SCCs follow the the topological order of $G'$
(3) We then show that the sum of changes $||PR-PR^-||_1$ is no more than $\epsilon$.
(4) We finally build the connection between $PR_k-PR_k^-$ and $PR_k^+-PR_k$ such that  $PR_k^+-PR_k=d\cdot M_{kk}^T\cdot(PR_k-PR_k^-)$, where $M_{kk}$ is the transition submatrix of nodes in the $k$-th SCC.
} %%%% eat



\stitle{Analysis of the block-wise algorithm}. While similar block-wise algorithms were originally proposed for Web graphs~\cite{Berkhin05}, we next show that they are even better for the TWPageRank computation associated with scholarly data.
%
The block-wise graph and its topological order can be done in $O(|V|+|E|)$ time~\cite{CormenLRS01}, and updating TWPageRank scores takes $O(|V|+|E_a|+t|E_i|)$ time as the edges in $E_a$ are only scanned once.
From these, we have the following.



\begin{lemma}
\label{prop-venue-time-complexity}
Given a citation graph or venue graph $G(V, E)$, algorithm \twprscc runs in  $O(|V|+|E_a|+t|E_i|)$ time, where $t$ is  the maximum number of iterations among all \sccs.
\end{lemma}



Recall that $t$ is very likely to be in the scale of tens to hundreds~\cite{Brin98:PageRank}.
%, and (b) $O(|V|+|E_a|+t|E_i|)$ is essentially $O(|V|+|E|+(t-1)|E_i|)$ as $|E_i|$ + $|E_a|$  = $|E|$.
Hence, the efficiency of block-wise algorithm \twprscc is mainly affected by $|E_i|$, \ie the smaller $|E_i|$ is, the faster algorithm \twprscc is.


%%%%%%%%%%%%%%%%%%%%%%%%%%%%%%%%%%%%%%%%%%%%%%%%%%
\begin{table}[tb!]
%\vspace{-2ex}
\begin{center}
\caption{\small Statistics of citation/venue graphs and Web graphs~\cite{LeskovecLDM09}}
\label{tab-batch}
%\begin{small}
\vspace{-.5ex}
\begin{tabular}{|c|c|c|c|c|}
\hline
 &   &  & \multicolumn{1}{c|}{\bf Largest }   & \multicolumn{1}{c|}{\bf \scc }    \\
\raisebox{1ex}[0pt]{\bf Graphs}      & \multicolumn{1}{c|}{\raisebox{1ex}[0pt]{\bf Nodes}} & \multicolumn{1}{c|}{\raisebox{1ex}[0pt]{\bf Edges}} &
\multicolumn{1}{c|}{\bf $|\scc|$} &  \multicolumn{1}{c|}{\bf edge ratio}    \\
\hline \hline
\aan  & 18,041/565 & 82.9K/22.5K &  20/18  & 0.9\%/2.8\%      \\  %\hline
\aminer  & 3.14M/56K & 14.3M/7.1M   & 23/1.5 &  1.6\%/2.1\%     \\ %\hline    % 1.5K = 1467
\magdata  & 127M/584K & 526M/162M & 351/10K & 0.1\%/1.8\%    \\ \hline
web-BS  &  685,230 & 7,600,595 & \multicolumn{1}{c|}{334,857} & \multicolumn{1}{c|}{59.51\%}\\  %\hline
web-G  & 875,713 & 5,105,039 & \multicolumn{1}{c|}{434,818} & \multicolumn{1}{c|}{66.98\%} \\  %\hline
\hline
\end{tabular}
%\end{small}
\end{center}
\vspace{-5ex}
\end{table}
%%%%%%%%%%%%%%%%%%%


\stitle{Observation.} It is well-known that article citations obey a natural temporal order, \ie an article only cites those published earlier, and it is really rare for the mutual citations between two articles published in the same time. That is, $|E_i|$ is essentially small for the citation graphs of scholarly articles.

To verify this observation, we also collect the statistics of citation graphs, venue graphs and Web graphs (shown in Table~\ref{tab-batch}), where Web graphs are extracted from berkely.edu and stanford.edu domains in 2002 and from the Google programming contest in 2002, respectively~\cite{LeskovecLDM09}.
Due to the existence of the ``bow tie'' structure and the giant \scc in Web graphs~\cite{BroderKMRRSTW00}, the edge ratios $|E_i|/|E|$ are greater than 59\% for Web graphs. In contrast, the \sccs in citation and venue graphs are quite small as a result of the temporal order of citation, and $|E_i|/|E|$ is less than 3\% for all tested citation and venue graphs.
%
This specific structure in scholarly data has been ignored in the literature, and has a positive impact on computations. Taking $t=100$ for example, algorithm \twprscc needs to scan $3|E|$ and $4|E|$ edges on citation and venue graphs, respectively, but over $59|E|$ edges on Web graphs.


By Corollary \ref{prop-prscc}, Lemma \ref{prop-venue-time-complexity} and the  above analysis of our block-wise algorithm, we have informally established Claim~\ref{thm-prestige}.


\eat{
In the Web graph with the ``bow tie" structure~\cite{BroderKMRRSTW00}, the nodes in the largest SCC amount to 51\% of the entire graph in 2014~\cite{MeuselVLB14}, and $|E_i|$ is already 25\% of $|E|$ even when edges are randomly generated.
%
However, the situation is completely different on scholarly data obeying temporal order in the sense that an article can only cite the ones published earlier. Moreover, it is pretty rare that two articles published in the same time mutually cite each other. Hence, both the citation graph and the venue graph can be treated as almost-DAGs~\cite{ObjectRank04}.
%
Indeed, on (\aan, \aminer, \magdata), the nodes in the largest SCC amount to (0.11\%, 0.00073\%, 0.00028\%) of the entire citation graph, and $|E_i|$ is (0.92\%, 0.14\%, 0.063\%) of $|E|$, respectively. %Here we present statistics of citation graphs since citation graphs are typically much larger than corresponding venue graphs and occupy most computation.
}



%Note that $O(|V|+|E_b|+t |E_w|)$ remains linear to $|E|$ in practice since $|E_w|$ is typically much smaller than $|E|$ (\eg 0.92\%) for scholarly data obeying a temporal order.



%\stitle{Temporal order}. Scholarly data follows temporal order in the sense that an article can only cite the articles published earlier, and, moreover, it is pretty rare that two articles published in the same time mutually cite each other. Hence, both the citation graph and the venue graph can be treated as a almost directed acyclic graph (DAG)~\cite{ObjectRank04}. Indeed, they become DAGs by replacing their strongly connected components (SCCs) with single nodes~\cite{CormenLRS01}.

%More specifically, the citation graph and the venue graph have the following two characteristics:
%a) the size of the largest SCC is much less than the one of the entire graph,
%and, as the result of that, b) the number of edges inside SCCs is much less than the total number of edges.



%\subsubsection{Prestige of authors}
%\label{subsubsec-prs-A}
%\subsubtitle{Prestige of authors}.






%\subsection{Time \& Space Complexity Analyses}

\stitle{Time \& space complexity analyses of the batch algorithm}.
%
By Lemma~\ref{prop-venue-time-complexity} and the analyses in Section~\ref{subsec-bat-alg}, one can verify that algorithm \batensemble takes $O(|V^c|+|V^v|+|E^c_a|+|E^v_a|+t|E^c_i|+t|E^v_i|+|PA|)$ time, where $|PA|$ is the total number of author-article relationships, and takes $O(7|V^c|+2|V^c{'}|+4|V^v|+2|V^v{'}|+|E^c|+|E^c{'}|+|E^v|+|E^v{'}|+2|A|+|D|)$ space, where $|A|$ is the number of authors and $|D|$ is the size of academic graph data $D$.
% TWPageRank on V^c: 3|V^c| + |E^c| + 2|V^c{'}| + |E^c{'}|
% TWPageRank on V^v: 3|V^v| + |E^v| + 2|V^v{'}| + |E^v{'}|
% Prestige of authors: |A|
% popularity: |V^c| + |V^v| + |A|
% Importance scores: 3|V^c|
% D: academic graph data, i.e., |PA|, |PV|, paper years, ect
% total: 7|V^c| + |E^c| + 2|V^c{'}| + |E^c{'}| + 4|V^v| + |E^v| + 2|V^v{'}| + |E^v{'}| + 2|A| + |D|

The key of algorithm \batensemble is to compute TWPageRank with algorithm \twprscc.
Compared with the traditional power method~\cite{Brin98:PageRank}, our block-wise algorithm \twprscc uses $(2|V^c{'}|+|E^c{'}|)$ and $(2|V^v{'}|+|E^v{'}|)$ extra space to store the block-wise graphs $G^c{'}(V^c{'},E^c{'})$ of $G^c$ and $G^v{'}(V^v{'},E^v{'})$ of $G^v$ and their topological orders, while speeds up computation by $O((t-1)(|E^c_a|+|E^v_a|))$.

%. That is, \batensemble speeds up computation with a reasonable extra space.







%%%%%%%%%%%%%%%%%%%%%%%%%%%%%%%% original ensemble version
\eat{
\subsection{Citation Ensemble Computation}
\label{subsec-citation}

%%% computation overview
We first present the computation of the citation ensemble.

It is easy to see that the popularity of articles can be computed by scanning all citations once and adding the freshness of citations to their corresponding articles,
by Eq.~(\ref{eq-pop}).
%
The prestige of articles is computed with TWPageRank on citation graphs, which occupies most computation cost of the citation ensemble.
Nevertheless, we speed-up  TWPageRank computation on citation graphs by exploiting the temporal order of scholarly data.

The main result here is stated as follows.

\begin{theorem}
\label{thm-citation}
TWPageRank on a citation graph $G^c(V^c, E^c)$ can be computed in $O(|V^c|+|E^c|)$ time.
\end{theorem}


%%% citation graph with temporal order --> the rare case of loops --> approximately a DAG
\stitle{Topological ordering}. The edges of the citation graph strictly follow temporal order, in the sense that an article can only cite the articles published earlier, and, moreover, it is pretty rare that two articles published in the same time mutually cite each other.
Hence, the citation graph $G^c$ can be treated as a directed acyclic graph with few side effects.
%
Accordingly, there exits a {\em topological order} for the nodes in a citation graph, \ie a sequence of its nodes such that every edge is directed from earlier to later.







We next prove Theorem~\ref{thm-citation} by providing such an algorithm with the desired property, by using the temporal order on a citation graph.





%%%%%%%%%%%%%%%%%%%%%Algorithm
\begin{figure}[tb!]
%\vspace{2ex}
\begin{center}
{\small
\begin{minipage}{3.36in}
\myhrule \vspace{-2ex}
\mat{0ex}{
%%%%%%%%%%%%%%%%%%%
%{\bf Algorithm}~\twprdag($G^c$) \\ %\hspace{6ex}/* $G$ is a DAG */
{\sl Input:\/} \= citation graph $G^c(V^c,E^c)$.\\
{\sl Output:\/} the TWPageRank vector of $G^c$. \\
\bcc \hspace{1.5ex}\=  $O$ := the topological order of $G^c$; \\
\icc\>  \For each node $v$ following $O$ \Do\\
\icc\>\hspace{2.5ex}\=  update $PR(v)$ using Eq.~(\ref{eq-twpr}); \\
\icc\> \Return $PR$.}
\vspace{-3ex} \myhrule
\end{minipage}
}
\end{center}
\vspace{-3ex}
\caption{\small Algorithm for TWPageRank on citation graphs} \label{alg-TWPageRank-citation}
\vspace{-3ex}
\end{figure}
%%%%%%%%%%%%Algorithm






\stitle{Algorithm \twprdag}. The algorithm is presented in Fig.~\ref{alg-TWPageRank-citation}, which takes as input a citation graph $G^c$ and returns its corresponding TWPageRank vector $PR$. It first computes a topological order of $G^c$ since $G^c$ is acyclic (line 1).
By the topological order with nodes in higher oder first, it updates the TWPageRank score of each node with the update rule in Eq.~(\ref{eq-twpr}) (lines 2--3), and finally  it returns the computed TWPageRank vector (line 4).
%such that node $v$ occurs iff every node $u$ that points to $v$, \ie $(u,v)\in E$, has already occurred.

%%% algorithm analysis: correctness & complexity

We now show that \twprdag is indeed the desired algorithm.

\begin{lemma}
\label{prop-nonitercomputing}
The TWPageRank vector $PR$ returned by \twprdag equals to the convergent TWPageRank vector $PR^*$.
\end{lemma}

\begin{proofSketch}
We have $PR^* = d M^T PR^* + \frac{1-d}{n} e$, as $PR^*$ is the convergent TWPageRank vector. Hence, we also have
\begin{small}
\begin{equation}
PR^*(v)=d \sum_{(u,v)\in E^c} M_{u,v} PR^*(u) + \frac{1-d}{n}.
\end{equation}
\end{small}

\vspace{-1ex}
Consider a topological order $v_1/\dots/v_n$ on citation graph $G^c(V^c,$ $E^c)$, we then prove $PR(v_k)=PR^*(v_k)$ ($k\in[1,n]$) by induction, from which we have the conclusion (see~\cite {ERank-full} for details).
\end{proofSketch}


\begin{lemma}
\label{prop-citation-time-complexity}
Given a citation graph $G^c(V^c, E^c)$, algorithm \twprdag runs in $O(|V^c|+|E^c|)$ time.
\end{lemma}


\begin{proof}
The topological order of citation graph $G^c$ can be computed in  $O(|V^c|+|E^c|)$ time~\cite{CormenLRS01} (line 1), and the updating step (lines 2--3) can be done in $O(|V^c|+|E^c|)$ time as well.
\end{proof}




%\ref{prop-converg},
By Lemmas \ref{prop-nonitercomputing} \& \ref{prop-citation-time-complexity}, we have shown Theorem~\ref{thm-citation}.

\stitle{Remark}. Observe that by employing the topological order, the time complexity of TWPageRank on citation graphs is significantly reduced to $O(|V^c|+|E^c|)$, instead of $O(t(|V^c|+|E^c|))$ that is closely related to the number $t$ of iterations.


\eat{
In this case the transition matrix $M$ of $G$ is organized as follows:
\begin{equation}
M=\left(
\begin{array}{ccccc}
\textbf{0} &  & \cdots &  &  \textbf{0} \\
M_{10}     &  \textbf{0} &  &  &  \\
M_{20}     &  M_{21}   &  \ddots &      & \vdots \\
\cdots     &        &       & \textbf{0} &  \\
M_{L0}     &    M_{L1}   &   \cdots   &     M_{L(L-1)}      & \textbf{0}  \\
\end{array}
\right) ,
\end{equation}
where $M_{ij}$ represents the transition between $V_i$ and $V_j$ and $\textbf{0}$ denotes all-zero matrices of proper sizes. Hence $M$ is a strictly lower triangular matrix, and we have:
}

\subsection{Venue Ensemble Computation}
\label{subsec-venue}






%%% computation overview
We next present the computation of the venue ensemble.
The popularity of venues in a specific year, is computed by averaging the popularity of the corresponding articles (given by the citation ensemble) published in the venues.

The prestige of venues in a specific year  is also computed by TWPageRank on a venue graph  $G^v(V^v, E^v)$. However, mutual citations are common on the venue graphs, \eg journal articles in a specific year may be published in different issues and/or numbers, and, hence, two journal venues in the same year can cite each other. That is, the venue graph is cyclic, and the linear algorithm on citation graphs does not work on venue graphs.

Nevertheless, articles typically cite those closely related articles, \eg from the same field of study. That is, there is a possibility of grouping venues.
It is known that a graph becomes acyclic by replacing its strongly connected components (SCCs) with single nodes~\cite{CormenLRS01}.
This motivates us to revise the algorithm by processing an SCC, instead of a single node, at each time, by which we can exploit the temporal order of scholarly data again.
%
More specifically, the edges of venue graph $G^v$ is first partitioned into two disjoint sets $E^v_w$ and $E^v_b$ with $E^v=E^v_w \cup E^v_b$, where $E^v_w$ and $E^v_b$ are sets of edges inside and across SCCs, respectively. Further, the update rule is revised by treating $E^v_b$ and $E^v_w$ differently as follows.
\begin{small}
\begin{equation}\label{eq-prscc}
PR(v) =  d \sum_{(u,v)\in E^v_w} M_{u,v} PR(u)
  + d \sum_{(u,v)\in E^v_b} M_{u,v} PR(u) +  \frac{1-d}{n}.
\end{equation}
\end{small}


\vspace{-2ex}
The main result here is stated as follows.

\begin{theorem}
\label{thm-venue}
TWPageRank on a venue graph $G^v(V^v, E^v)$ can be done in  $O(|V^v|+|E^v_b|+t|E^v_w|)$ time.
\end{theorem}


We next prove Theorem~\ref{thm-venue} by providing such an algorithm with the desired property, which uses the temporal order, and
processes an SCC, instead of a single node, at each time on a venue graph.


%%% algorithm details
%%%%%%%%%%%%%%%%%%%%%Algorithm
\begin{figure}[tb!]
%\vspace{2ex}
\begin{center}
{\small
\begin{minipage}{3.36in}
\myhrule \vspace{-2ex}
\mat{0ex}{
%%%%%%%%%%%%%%%%%%%
%{\bf Algorithm}~\twprscc($G^v$, $\epsilon$) \\
{\sl Input:\/} \= venue graph $G^v(V^v,E^v)$, iteration threshold $\epsilon$.\\
{\sl Output:\/} the TWPageRank vector of $G^v$. \\
\bcc \hspace{1.5ex}\=  $G'$ := the converted graph of $G^v$; \\
\icc \>  $O$ := topological order of $G'$; \\
\icc\>  \For each node $v'$ following $O$ \Do\\
\icc\>\hspace{2.5ex}\=  $scc$ := the corresponding SCC of $v'$; \\
\icc\>\> \While sum of changes of $PR(v)$ where $v\in scc$ exceeds $\frac{|scc|}{|V^v|} \epsilon$ \Do \\
\icc\>\>\hspace{3.5ex}\= update $PR(v)$ using Eq.~(\ref{eq-prscc}) where $v\in scc$; \\
\icc\> \Return $PR$.}
\vspace{-3ex} \myhrule
\end{minipage}
}
\end{center}
\vspace{-3ex}
\caption{\small Algorithm for TWPageRank on venue graphs} \label{alg-TWPageRank-venue}
\vspace{-3ex}
\end{figure}

\eat{%%%%%%%%%%%%%%
{\bf Algorithm}~\batensemble\\
\sttab {\sl Input:\/} \= academic graph data $D$, iteration threshold $\epsilon$.\\
{\sl Output:\/} scholarly article ranking $R$ of $D$. \\
%%% \icc \hspace{1.5ex}\= extract article (venue) citation graph $G_c$ ($G_v$) from D; \\
\icc \hspace{1.5ex}\= compute prestige of citation ensemble with \twprdag($G_c$); \\
\icc\> compute prestige of venue ensemble with \twprscc($G_v$, $\epsilon$);\\
\icc\> compute prestige of author ensemble based on citation ensemble;\\
\icc\> compute popularity of three ensembles;\\
\icc\> \Return ranking $R$ based on three ensembles;}

%%%%%%%%%%%%Algorithm


\stitle{Algorithm~\twprscc}.
The SCC-based algorithm to compute the TWPageRank vector for venue graph $G^v$ is shown in Fig.~\ref{alg-TWPageRank-venue}.

It first computes the converted graph $G'$  by treating SCCs in $G^v$ as single nodes (line 1), and then derives a topological order $O$ of $G'$ (line 2). It then iteratively updates the TWPageRank scores of nodes in each SCC in the topological order of $G'$ using  Eq.~(\ref{eq-prscc}) (lines 3--6), and finally returns the TWPageRank vector (line 7). More specifically, the iteration continues until the sum of TWPageRank score changes is less than $\frac{|scc|}{|V^v|} \epsilon$ (lines 5--6).




\begin{lemma} \label{prop-prscc}
The TWPageRank vector $PR$ returned by~\twprscc converges such that $||PR-PR^*||_1 < \epsilon$, where $PR^*$ is the convergent TWPageRank vector.
\end{lemma}

\begin{proofSketch}
%We first prove that $||PR^+-PR||_1 < \epsilon$ where $PR^+=d\cdot M^T\cdot PR + \frac{1-d}{n}\cdot e$.
%We then prove that $||PR^*-PR||_1 < ||PR^+-PR||_1$.
We first prove that the sum of changes after another iteration from $PR$, \ie $||d M^T PR + \frac{1-d}{n} e-PR||_1$, is smaller than $\epsilon$, and then prove that $||PR^*-PR||_1$ is smaller than the sum of changes (see~\cite {ERank-full} for details). % since $\epsilon > ||PR^+-PR||_1 = ||d\cdot M^T \cdot (PR-PR^*)||_1 + ||PR-PR^*||_1$.
\end{proofSketch}

%%% detailed proof sketch
\eat{
(1) We first prove that the sum of changes of TWPageRank vector after another iteration is less than $\epsilon$, \ie $||PR^+-PR||_1 < \epsilon$ where $PR^+=d\cdot M^T\cdot PR + \frac{1-d}{n}\cdot e$.
(2) Let $PR^-$ be the TWPageRank vector in the previous iteration of each SCC, and $PR_k$, $PR_k^-$ and $PR_k^+$ be the TWPageRank vectors of nodes within the $k$-th SCC in $PR$, $PR^-$ and $PR^+$, respectively. Note that SCCs are ordered such that the sequence of the corresponding nodes of SCCs follow the the topological order of $G'$
(3) We then show that the sum of changes $||PR-PR^-||_1$ is no more than $\epsilon$.
(4) We finally build the connection between $PR_k-PR_k^-$ and $PR_k^+-PR_k$ such that  $PR_k^+-PR_k=d\cdot M_{kk}^T\cdot(PR_k-PR_k^-)$, where $M_{kk}$ is the transition submatrix of nodes in the $k$-th SCC.
} %%%% eat

\begin{lemma}
\label{prop-venue-time-complexity}
Given a venue graph $G^v(V^v, E^v)$, algorithm \twprscc runs in  $O(|V^v|+|E^v_b|+t|E^v_w|)$ time, where $t$ is  the maximum number of iterations among all SCCs.
\end{lemma}

\begin{proof}
Observe the following. (1) The converted graph $G'$ and topological order $O$ can be computed in $O(|V^v|+|E^v|)$~\cite{CormenLRS01} (lines 1,2), and (2) updating the TWPageRank scores takes $O(|V^v|+|E^v_b|+t |E^v_w|)$ where edges in $|E^v_b|$ are only scanned once (lines 3--6). Note that $|E^v_w|$ is typically much smaller than $|E^v|$ for scholarly article data obeying a temporal citation order in practice.
\end{proof}

By Lemmas \ref{prop-prscc} \& \ref{prop-venue-time-complexity}, we have shown Theorem~\ref{thm-venue}.




\subsection{Author Ensemble Computation}

As the prestige (popularity) of authors is defined as the average prestige (popularity) of their published articles, it suffices to scan through all author-article  relationships for computing both the prestige and the popularity of authors.

\subsection{The Complete Batch Algorithm}
\label{subsec-bat-alg}
We finally present the complete batch algorithm \batensemble, which combines the computation of the citation, author and venue ensembles with Eq.~(\ref{eq-ensemble}). It takes as input academic graph data $D$ and an iteration threshold $\epsilon$ and returns the scholarly article ranking of $D$. It first constructs the citation and venue graphs $G^c(V^c,E^c)$ and $G^v(V^v,E^v)$. Then it computes the prestige and popularity of citation, venue and author ensembles.
Finally, it combines the prestige and popularity of three ensembles to produce the final ranking.

\stitle{Time \& space complexity analyses}. By the analyses above and Lemmas~\ref{prop-citation-time-complexity} \& \ref{prop-venue-time-complexity}, it is easy to verify that algorithm \batensemble runs in $O(|V^c|+|E^c|+|V^v|+|E^v_b|+t|E^v_w|+|PA|)$, where $|E^v_w|$ is the number of edges inside SCCs of $G^v$ and $|PA|$
is the total number of author-article relationships.

%Observe the following. (1) Procedures \twprdag and \twprscc run in $O(|V_c|+|E_c|)$ and $O(|V_v|+|E_v|+t\cdot|E_v^w|)$, respectively (lines 12--13). (2) Computing prestige of author ensemble costs $O(|PA|)$ time (line 14). (3) Computing popularity of three ensembles costs $O(|V_c|+|E_c|+|PA|)$ time (line 15). Finally, (4) generating ranking $R$ costs $(|V_c|+|PA|)$ time.

Compared with the traditional power method, algorithm \batensemble uses (a)  $|V^c|$ extra space to store the topological order of $G^c$, (b) $(2|V'|+|E'|)$ extra space to store the converted graph $G'(V',E')$ of $G^v$ and its topological order, and (c) $|V^c|$ space is saved for the TWPageRank vector of $G^c$. To conclude, algorithm \batensemble only uses limited extra space to speed-up computation.


%\subsection{Author Ensemble}
%\label{subsec-author}

%\subsection{Optimization Techniques}
%\label{subsec-opt}

%%%% dangling node manipulation
\eat{
The article citation graph has a strong possibility to contain dangling nodes which do not have any out-links, \ie citations, due to factors such as dataset coverage and data missing.
These dangling nodes are usually handled by adding artificial edges to all nodes in the graph such that its prestige is propagated to all node. For Web pages this can be interpreted as users randomly select a new page to visit when current one does not contain any out-link to follow. However, dangling articles are more likely because of missing citations. It is inappropriate to propagate the prestige of dangling articles to others which will erroneously increase the prestige of most articles. Hence, we do not manipulate the dangling nodes on the article citation graph.
The relationship between the two PageRank vectors with or without manipulating dangling nodes is given by Proposition~\ref{prop-prvector}:
\begin{prop} \label{prop-prvector}
PageRank vectors with or without manipulating dangling nodes can derive the other through scaling.
\end{prop}
}

%%% proof of citation graph is a DAG
\eat{
\begin{proof}
Suppose there exists a circle $v_1/v_2/\dots/v_k/v_1$ in $G$ (where $k>1$). Edges in the circle imply $T_{v_1}<T_{v_2}<\dots <T_{v_k}<T_{v_1}$, where $T_{v_i}<T_{v_j}$ means $T_{v_i}$ is strictly earlier than $T_{v_j}$. However, this cannot be true since there does not exist $T_{v_k}$ which is both strictly earlier and later than $T_{v_1}$ in the same time. Hence, $G$ is acyclic and further a DAG since $G$ is directed.
\end{proof}
} %%%%eat

}
