% IEEE Paper Template for US-LETTER Page Size (V1)
% Sample Conference Paper using IEEE LaTeX style file for US-LETTER pagesize.
% Copyright (C) 2006-2008 Causal Productions Pty Ltd.
% Permission is granted to distribute and revise this file provided that
% this header remains intact.
%
% REVISION HISTORY
% 20080211 changed some space characters in the title-author block
%
\documentclass[10pt,conference,letterpaper]{IEEEtran}
\usepackage{times,amsmath,epsfig}

%jing
\usepackage{graphicx}
\usepackage{balance}  % for  \balance command ON LAST PAGE  (only there!)
%
\usepackage{amsmath,amssymb}
%\usepackage[ruled]{algorithm2e}
\usepackage{xcolor}
\usepackage{caption2}
%\usepackage{verbatim}
\usepackage{url}
%%wj\usepackage{paralist}
%%wj\usepackage{lstautogobble}
%%wj\usepackage{listings}
%
%\usepackage{subcaption}
\usepackage{amsmath}
%\usepackage{bbold}
\usepackage{subfigure}
%\usepackage{enumitem}
\usepackage{textcomp}
%\usepackage{graphicx}
\usepackage{multirow}
\usepackage{comment}

%%wj
\begin{comment}
\lstset{%
	basicstyle=\ttfamily\small,
	identifierstyle=\rmfamily,
	keywordstyle=\rmfamily\bfseries,
	numbers=left,
	numberstyle=\footnotesize,
	numbersep=5pt,
	tabsize=3,
	frame=single,
	breaklines=true,
	breakatwhitespace=false,
	breakautoindent=true,
	mathescape=true,
	autogobble=true,
	linewidth=\linewidth,
	escapechar=@,
}
\end{comment}
%$wj

\usepackage{algorithm}
\usepackage{algpseudocode}
\usepackage{booktabs}

%%jing


%
%\title{\sys{}: Modeling the Retweeting Behaviors of User Groups in Social Media}
%\title{Grouping Users for Modeling Retweeting Behaviors}
\title{Incorporating User Grouping into \Retg{} Behavior Modeling}
% retweet or what?

%
\author{%
% author names are typeset in 11pt, which is the default size in the author block
{First Author{\small $~^{\#1}$}, Second Author{\small $~^{*2}$}, Third Author{\small $~^{\#3}$} }%
% add some space between author names and affils
\vspace{1.6mm}\\
\fontsize{10}{10}\selectfont\itshape
% 20080211 CAUSAL PRODUCTIONS
% separate superscript on following line from affiliation using narrow space
$^{\#}$\,First-Third Department, First-Third University\\
Address Including Country Name\\
\fontsize{9}{9}\selectfont\ttfamily\upshape
%
% 20080211 CAUSAL PRODUCTIONS
% in the following email addresses, separate the superscript from the email address 
% using a narrow space \,
% the reason is that Acrobat Reader has an option to auto-detect urls and email
% addresses, and make them 'hot'.  Without a narrow space, the superscript is included
% in the email address and corrupts it.
% Also, removed ~ from pre-superscript since it does not seem to serve any purpose
$^{1}$\,first.author@first-third.edu\\
$^{3}$\,third.author@first-third.edu%
% add some space between email and affil
\vspace{1.2mm}\\
\fontsize{10}{10}\selectfont\rmfamily\itshape
% 20080211 CAUSAL PRODUCTIONS
% separated superscript on following line from affiliation using narrow space \,
$^{*}$\,Second Company\\
Address Including Country Name\\
\fontsize{9}{9}\selectfont\ttfamily\upshape
% 20080211 CAUSAL PRODUCTIONS
% removed ~ from pre-superscript since it does not seem to serve any purpose
$^{2}$\,second.author@second.com
}
%



\begin{document}

%jing
\newtheorem{definition}{Definition}
\newtheorem{problem}{Problem}
\newtheorem{theorem}{Theorem}
\newtheorem{remark}{Remark}
\newtheorem{lemma}[theorem]{Lemma}
\newtheorem{corollary}[theorem]{Corollary}
%
\newcommand{\sys}{\textcolor{blue}{SYSNAME}}
\newcommand{\tbc}{\textcolor{blue}{[To Be Completed]}}
\newcommand{\retg}{\textcolor{blue}{retweeting}}
\newcommand{\retgs}{\textcolor{blue}{retweetings}}
\newcommand{\Retg}{\textcolor{blue}{Retweeting}}
\newcommand{\retd}{\textcolor{blue}{retweeted}}
\newcommand{\ret}{\textcolor{blue}{retweet}}
%%jing






\maketitle



\begin{abstract} 
Social media applications are emerging, with rapidly growing users and large numbers of \retd{} blogs every minute. 
The variety among users makes it difficult to model their \retg{} activities.
Obviously, it is not suitable to cover the overall users by a single model.
Meanwhile, building one model per user is not practical. 
To this end, this paper presents a novel solution, of which the principle is to model the \retg{} behavior over user groups.
Our approach, \sys{}, consists of three key components for extracting user based features, clustering users into groups, and modeling upon each group.
Particularly, we look into the user interest from different perspectives including long-term/recent interests and explicit/implicit interests, which results deep analyses towards the \retg{} behavior and proper models in the end.
We have evaluated the performance of \sys{} by datasets of real-world social networking applications and a number of query workloads, showcasing its benefits.     
\end{abstract}
% behavior and activity in this paper are interchangeable

\input{01-intro.tex}
\section{\sys{} Overview}
\label{sec:overv}

\subsection{Problem Formulation}
% {user, blog, user-blog} => model
% query (a user u, a blog b that is created or forwarded by u's friend)
% returns: Y/N u shall forward b 

We consider people's \retg{} behavior in social media.
For simplicity, with a given user, we assume that blogs created or \retd{} by his/her followees cover the overall candidates, from which the said user may \ret{}.
All our results could straightforwardly generalize to alternative candidate scopes.
%e.g., the like/unlike behavior against the candidate of remarking [double check the network language; register a Facebook]
%e.g., positive comments in the overall comments

\begin{definition}
\label{def:blog}
A blog $B = (O, T, M, R)$ consists of the owner $O$ (a.k.a. user in this paper) to whom $B$ belongs (either created or \retd{}), the timestamp $T$ showing when $B$ is generated, the blog message $M$ and a bit $R$ denoting $B$ is \retd{} (1) or originally created (0) by the owner $O$.
\end{definition}

\begin{comment}
\begin{definition}
A blog $B = (O, T, M, C)$ consists of the owner $O$ to whom $B$ belongs (either created or \retd{}), the timestamp $T$ showing when $B$ is generated, the blog message $M$ and a set of counters $C_s(B) = \{\#comment,\ \#like,\ \#\ret{}\}$ regarding the number of being commented, liked, and \retd{}.
\end{definition}
\end{comment}

\begin{definition}
\label{def:user}
A user $U = (B_s, R_s, E_s)$ consists of three sets regarding the user's blogs $B_s$, followers $R_s$ and followees $E_s$ separately.
Each follower/followee per se refers to a user.
\end{definition}

The mapping between blog $B$ and user $U$ is a bilateral operation, i.e., $U = O(B)$ and $B \in B_s(U)$, through ID(s) of user and blog respectively.

Informally, providing a set of users $\{U\}$ and the associated blogs $\{B\}$, as well as a blog query $b$ and a follower of $O(b)$ written as $f$, i.e., $f \in R_s(O(b))$, \sys{} shall build a \retg{} model for $(\{U\},\ \{B\})$, upon which Y/N is returned regarding whether $f$ shall \ret{} $b$.


\subsection{\sys{} Framework}
\sys{} is designed from the ground up as a system for modeling users' \retg{} behavior in social media.
Figure \ref{fig:framework} shows the architectural components of \sys{}, comprising three subsystems: Data Storage, Processing Runtime and Profile Demonstrator.
 
\begin{figure}[!htb]
\centering
\includegraphics[width=.96\linewidth]{figures/architecture}
\caption{\sys{} Architecture}
\label{fig:framework}
\end{figure}


% #like and #comment are saved; could generalize \sys{} to model the liking behavior (among commented blogs)
\begin{comment}
\begin{figure}[!htb]
\centering
\includegraphics[width=.99\linewidth]{figures/microblog}
\caption{Blog Data in \sys{}}
\label{fig:blog}
\end{figure}
\end{comment}

\begin{comment}
\begin{figure}[!htb]
\centering
\includegraphics[width=.99\linewidth]{figures/user}
\caption{User Data in \sys{}}
\label{fig:user}
\end{figure}
\end{comment}


\textbf{Data Storage.} The underlying Data Storage subsystem stores data to be processed by \sys{}, i.e., data of blogs and users, as shown in \textit{Definition} \textit{\ref{def:blog}} and \textit{\ref{def:user}}.

\textbf{Processing Runtime.}
In the heart of \sys{} lies the Processing Runtime subsystem, which consists of three major components as follows.
\begin{enumerate}
	\item Feature Extractor: By coalescing the blog data, each user is depicted by a bunch of features, which are grouped into three categories. They are features of \textit{Info} (e.g., the number of followers and followees), \textit{Behavior} (e.g., the frequency and the popular slots of \retg{}) and \textit{Interest} (e.g., the long-term/recent interests, as well as the explicit/implicit interests). These features are extracted from the stored data by Feature Extractor and serve as the input of User Clusterer.
	\item User Clusterer: Providing the user-based features, User Clusterer takes charge of the clustering task such that each user falls into a proper group. 
	\item Group Modeler: For each group obtained by User Clusterer, Group Modeler employs both positive and negative samples to train a model, over which the testing of users' \retg{} behavior is performed.
\end{enumerate}
	
\textbf{Profile Demonstrator.} At the top layer of \sys{}, it is the Profile Demonstrator subsystem for visualization. 
For the time being, Profile Demonstrator presents \tbc{}.




\section{\sys{} Principles}
\label{sec:prin}





\section{\sys{} Algorithms and Structures}
\label{sec:algo}
\section{Performance Evaluation}
\label{sec:perf}
\section{Related Work}
\label{sec:rela}
\section{Conclusions}
\label{sec:conclu}
\input{acknow.tex}







\cite{IEEEexample:article_typical}

\balance

\bibliographystyle{IEEEtran}
\bibliography{IEEEabrv,icde-zhu}

\end{document}
