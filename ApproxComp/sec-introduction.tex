\section{Introduction}
\label{sec-intro}


Over the past a few years, research and development has made significant progresses on big data analytics with the supports from both governments and industries all over the world, such as Spark\footnote{\small \url{https://spark.apache.org}}, IBM Watson\footnote{\small \url{https://www.ibm.com/watson}} and Google AlphaGo\footnote{\small \url{https://deepmind.com/research/alphago}}. A fundamental issue for big data analytics is the efficiency, and various advances towards attacking this issues have been achieved recently, from theory to algorithms to platforms~\cite{FanGN13,Jordan15,ZahariaXWDADMRV16}.  In this article, we shall present the idea of two classes of techniques for efficient and effective big data analytics: query approximation and data approximation, based on our recent research experiences. Different from existing approximation techniques~\cite{CormenLRS01,approx03,FanH14}, the approximation computation that we are going to introduce is not unnecessarily theoretically bounded with respect to optimal solutions, but asks for both efficiency and effectiveness in practice.


\section{Key Insights}
\label{sec-insight}

We systematically introduce approximate computation (\ie query and data approximation) for big data analytics, targeting at both efficiency and effectiveness in practice.

We explain the idea and rational of query approximation,  and show efficiency can be obtained with high effectiveness in practice by its embodiment for graph pattern matching, trajectory compression and dense subgraph computation.


We explain the idea and rational of data approximation,  and show efficiency can be obtained even without the sacrifice of effectiveness in practice by its embodiment for shortest paths/distances and link prediction.



\eat{
Contributed Articles cover the wide and abundant spectrum of the computing field��its open challenges, technical visions and perspectives, educational aspects, societal impact, significant applications and research results of high significance and broad interest. Following the roots of Communications, these submissions are peer-reviewed to ensure the highest quality. Topics covered must reach out to a very broad technical audience. While articles appearing in an ACM Transactions journal are aimed at a specialized audience, articles in Communications should be aimed at the broad computing and information technology community.

A Contributed Article should set the background and provide introductory references, define fundamental concepts, compare alternate approaches, and explain the significance or application of a particular technology or result by means of well-reasoned text and pertinent graphical material. The use of sidebars to illustrate significant points is encouraged.

Full-length Contributed Articles should consist of up to 4,000 words, contain no more than 25 references, 3-4 tables, 3-4 figures, and be submitted to: http://cacm.acm.org/submissions.

Submissions to the Contributed Articles section should be accompanied by a cover letter indicating:

? Title and the central theme of the article;
? Statement addressing why the material is important to the computing field and of value to the Communications reader; and,
? Names and email addresses of three or more recognized experts who would be considered appropriate to review the submission.

%https://cacm.acm.org/about-communications/author-center/author-guidelines


%Lionel M. Ni
%Elisa Bertino
%Yannis Ioannidis

}
