\section{Beyond Query and Data Approximations}
\label{sec-beyond}

There exist no single techniques fit all for big data analytics, and  it is often necessary to combine different techniques to obtain satisfiable solutions. 
We have seen that data sampling and data compression techniques help to achieve a balance between efficiency and effectiveness. 
We next roughly introduce three other potential techniques used for big data analytics. 


When there are data updates, query answers typically need to be re-computed to reflect the changes. Incremental computation is a technique that attempts to reduce computations by reusing previous computing efforts and only computing those answers that ``depend on'' the changed data.


 incremental algorithms have been developed for
various applications.  Thomas W. Reps has done pioneering work on the study of incremental computation, and he observed that
the complexity of incremental algorithms was more accurately characterized
in terms of the size of the area affected by the updates, rather than the size
of the entire input~\cite{Reps96}.

One may notice that incremental algorithms can be utilized to achieve query approximation.



Data partitioning, Data indexing and Distributed computing~\cite{MaLHLH16} can also be combined to design query approximation and data approximation approaches for big data analytics.



