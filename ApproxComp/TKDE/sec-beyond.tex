\section{Beyond Approximation Techniques}
\label{sec-beyond}

  For big data analytics, there are no one-size-fits-all techniques, and it is often necessary to combine different techniques to obtain good solutions.


We have seen that sampling helps to achieve a balance between efficiency and effectiveness for approximate query processing~\cite{ChaudhuriDK17,Mozafari17,Kraska17,GarofalakisG01} and link prediction~\cite{DuanAMHH16,DuanMAMH17}},
and other techniques such as incremental computation~\cite{Reps96,FanLMTWW10,rankicde2018}, distributed computing~\cite{MaCHW12,FanXWYJZZCT17}, and system techniques~\eg caching~\cite{WangLMNT18}, hardware~\cite{AbergerLTNOR17,Han0Y18} can also be unitized for big data analytics, and can even be combined for designing query and data approximation techniques for big data analytics. 

It is worth pointing out that for all kind of techniques big data analytics, various computing resources should be seriously considered, \eg using bounded resources for approximation~\cite{CaoF17} and for incremental computation~\cite{abs-1801-01012}. It is also obvious that techniques beyond query approximation and data approximation, such as theory and systems~\cite{FanGN13,Jordan15,ZahariaXWDADMRV16}, are must things for big data analytics.


