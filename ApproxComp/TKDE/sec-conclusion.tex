\section{Conclusions}
\label{sec-conclusion}

In this article we have systematiclly introduced approximation computation techniques for efficient and effective big data analytics.
Furthermore, although approximate computation does not put
theoretical bounds with respect to optimal solutions, it does expect a balance between efficiency and effectiveness. Indeed, (a) our query approximation techniques~\cite{ShuaiMaVLDB12,tods-MaCFHW14,LinMZWH17,MaHWLH17} show that efficiency can be obtained with high accuracy in practice, and (b) our data approximation techniques~\cite{MaFLWCH16,MaFLWCH17,HuAMH16,DuanAMHH16,DuanMAMH17} show that efficiency and accuracy can be obtained simultaneously for certain data analytic tasks. That is, though approximate computation is for a situation where an approximate result is sufficient for a purpose, its design policy is not always to sacrifice effectiveness for efficiency.


%policy is not always to sacrifice effectiveness for efficiency.


%approximate computation  the design policy of is not always to sacrifice effectiveness for efficiency, but to achieve both efficiency and effectiveness in practice for a situation where an approximate result is sufficient for a purpose.

